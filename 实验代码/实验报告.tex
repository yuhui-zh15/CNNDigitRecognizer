% !Mode:: "TeX:UTF-8"
\documentclass{article}
\usepackage[hyperref, UTF8]{ctex}
\usepackage[dvipsnames]{xcolor}
\usepackage{geometry}
\usepackage{amsmath}
\usepackage{amsfonts}
\usepackage{listings}
\usepackage{pgfplotstable}
\usepackage{graphicx,float,wrapfig}
\usepackage{pgfplots}
\usepackage{fontspec}
\usepackage{hyperref}
\usepackage{booktabs} % 表格上的不同横线
\setmonofont[Mapping={}]{Monaco}  %英文引号之类的正常显示,相当于设置英文字体
\setsansfont{Monaco} %设置英文字体 Monaco, Consolas,  Fantasque Sans Mono
% \setmainfont{苹方-简} %设置英文字体

\newcommand{\chuhao}{\fontsize{42pt}{\baselineskip}\selectfont}
\newcommand{\xiaochuhao}{\fontsize{36pt}{\baselineskip}\selectfont}
\newcommand{\yihao}{\fontsize{28pt}{\baselineskip}\selectfont}
\newcommand{\erhao}{\fontsize{21pt}{\baselineskip}\selectfont}
\newcommand{\xiaoerhao}{\fontsize{18pt}{\baselineskip}\selectfont}
\newcommand{\sanhao}{\fontsize{15.75pt}{\baselineskip}\selectfont}
\newcommand{\sihao}{\fontsize{14pt}{\baselineskip}\selectfont}
\newcommand{\xiaosihao}{\fontsize{12pt}{\baselineskip}\selectfont}
\newcommand{\wuhao}{\fontsize{10.5pt}{\baselineskip}\selectfont}
\newcommand{\xiaowuhao}{\fontsize{9pt}{\baselineskip}\selectfont}
\newcommand{\liuhao}{\fontsize{7.875pt}{\baselineskip}\selectfont}
\newcommand{\shibahao}{\fontsize{18pt}{\baselineskip}\selectfont}
\newcommand{\shisihao}{\fontsize{14pt}{\baselineskip}\selectfont}
\newcommand{\qihao}{\fontsize{5.25pt}{\baselineskip}\selectfont}

\definecolor{mygreen}{rgb}{0,0.6,0}
\definecolor{mygray}{rgb}{0.5,0.5,0.5}
\definecolor{mymauve}{rgb}{0.58,0,0.82}
\lstset{ %
    backgroundcolor=\color{white},   % choose the background color
    basicstyle=\footnotesize\ttfamily,        % size of fonts used for the code
    columns=fullflexible,
    breaklines=true,                 % automatic line breaking only at whitespace
    captionpos=b,                    % sets the caption-position to bottom
    tabsize=4,
    backgroundcolor=\color[RGB]{245,245,244},            % 设定背景颜色
    commentstyle=\color{mygreen},    % comment style
    escapeinside={\%*}{*)},          % if you want to add LaTeX within your code
    keywordstyle=\color{blue},       % keyword style
    stringstyle=\color{mymauve}\ttfamily,     % string literal style
    showstringspaces=false,                % 不显示字符串中的空格
    frame=none,
    rulesepcolor=\color{red!20!green!20!blue!20},
    % identifierstyle=\color{red},
    language=python,
}

% 设置hyperlink的颜色
\newcommand\myshade{85}
\colorlet{mylinkcolor}{violet}
\colorlet{mycitecolor}{YellowOrange}
\colorlet{myurlcolor}{Aquamarine}

\hypersetup{
  linkcolor  = mylinkcolor!\myshade!black,
  citecolor  = mycitecolor!\myshade!black,
  urlcolor   = myurlcolor!\myshade!black,
  colorlinks = true,
}

\title{\Huge{人工智能导论实验报告}}

\date{}

\begin{document}
\author{张钰晖\\2015011372, yuhui-zh15@mails.tsinghua.edu.cn, 185-3888-2881}
\maketitle
\tableofcontents

\newpage
\section{问题描述}

日常生活中,我们经常需要在填写信件、银行开卡填写表单的时候手写大量的数字,如身份证号、手机号、邮编等等。事实上,专业人员在过去录入这些信息非常费时费力,而且还容易出现错误。现在请你设计一个手写体数字照片智能分类程序,省去人工识别0-9数字的麻烦。

在Kaggle竞赛平台上有Digit Recognizer https://www.kaggle.com/c/digit-recognizer的经典任务,并且提供了上万张手写体数字照片的灰度数据(可以借助Matlab还原照片)。目前,该任务已经被许多科研人员设计的人工智能程序完美解决,达到100\%测试精度。请大家在熟悉Kaggle平台的使用之后,直接通过该平台完成Digit Recognizer的竞赛任务,与一线科学家一较高下;并将你在从事竞赛过程中的心得体会、实验流程,以及最佳测评结果截图写入实验报告,同时在附录中递交你的代码。

\section{模型选择}
  根据Tensorflow官方教程中MNIST FOR EXPERT中的CNN部分,和Lenet-5的结构,以及相关论文文献,最终选择了模型。

  一般卷积神经网络均为2层卷积层,然后连接2-3层全连接层,中间层用Relu函数激活,输出层用Softmax函数激活。

  实验分别以两大类方案进行实验,并对每类方案进行了参数调节,两大类方案如下,参数调节见后文。


  \subsection{模型1(MNIST FOR EXPERT)}

  输入层:长度为784的向量,展平为28x28。(28x28)
  
  第一层:卷积层一。用32个卷积核进行Padding=SAME类型卷积,进行2*2最大池化,Relu函数激活。(32@14x14)
  
  第二层:卷积层二。用64个卷积核进行Padding=SAME类型卷积,进行2*2最大池化,Relu函数激活。(64@7x7)
  
  第三层:全连接层一。用1024个神经元与上一层进行全连接,Relu函数激活。(1024)
  
  输出层:全连接层二。用10个神经元与上一层进行全连接,Softmax函数激活。(10)
  
  \subsection{模型2(LeNet-5)}

  输入层:长度为784的向量,展平为28x28。(28x28)
  
  第一层:卷积层一。用6个卷积核进行Padding=VALID类型卷积,进行2*2最大池化,Relu函数激活。(6@12x12)
  
  第二层:卷积层二。用16个卷积核进行Padding=VALID类型卷积,进行2*2最大池化,Relu函数激活。(16@4x4)
  
  第三层:全连接层一。用120个神经元与上一层进行全连接,Relu函数激活。(120)

  第四层:全连接层二。用84个神经元与上一层进行全连接,Relu函数激活。(84)
  
  输出层:全连接层三。用10个神经元与上一层进行全连接,Softmax函数激活。(10)

  下图展示了图片大小为32x32时的LaNet-5神经网络
   \begin{figure}[H]
  \centering\includegraphics[width=4.5in]{./01.png}
  \caption{LaNet-5神经网络}
  \end{figure}


\section{细节实现}

作为开始,我们读进来所需要的数据。

train.csv为训练集文件,包含42000行和785列。每一行代表一张手写的数字图片,第一列为对应数字的标签,剩下列为784个像素灰度值(0\~255),对应28*28的图片大小。

test.csv为测试集文件,包含18000行和784列。每一行为784个像素灰度值(0\~255),对应28*28的图片大小。

将训练集样本部分抽取出来作为验证集,以检测神经网络训练结果。

至此我们得到了训练集、验证集和测试集。

下图展示了其中一个训练样本。

 \begin{figure}[H]
  \centering\includegraphics[width=1.0in]{./1.png}
  \caption{一个训练样本}
  \end{figure}

建立神经网络结构,进行卷积、池化。下图展示了卷积和池化的原理。
 \begin{figure}[H]
  \centering\includegraphics[width=4.5in]{./04.png}
  \caption{卷积}
  \centering\includegraphics[width=4.5in]{./03.png}
  \caption{卷积核}
  \centering\includegraphics[width=4.5in]{./02.png}
  \caption{池化}
  \end{figure}

定义交叉熵(Cross Entropy)后,用改进的梯度下降法(Adam Optimizer)去降低交叉熵,训练50000次,每批次(Batch)投入50个训练数据。

因为所用神经网络神经元非常多,通过训练曲线可以看出,50000次保证不至于欠拟合。同时为了防止过拟合,每5000次用验证集检验一次结果,如果结果优于上次结果,则保存本次结果,否则弃掉本次结果。

下图展示了某一次训练时的训练准确率曲线。

 \begin{figure}[H]
  \centering\includegraphics[width=4.5in]{./2.png}
  \caption{训练准确率曲线}
  \end{figure}

由训练曲线可以看出,该神经网络收敛非常快,在训练500次时验证集准确率便已经达到93.5\%,之后收敛逐渐变慢,最后趋于稳定。

最终可以看出,在训练集上,准确率为99.809\%,和实际提交结果99.143\%相似,略低的原因可能是测试集噪声比较大。

为了提高准确率,我们可以加入更多的训练元素,比如著名的MNIST样本集结构和本次任务图像结果基本完全类似,引入MNIST样本集进行训练,同样训练50000次,准确率为99.871\%,截止写实验报告时,排名第52。


\section{实验结果}

\subsection{模型的选择}

下文将对比模型1和模型2的结果,训练次数为50000次。

\begin{description}

\item[(1)]模型1(MNIST FOR EXPERT):

训练曲线如下:

 \begin{figure}[H]
  \centering\includegraphics[width=4.5in]{./2.png}
  \caption{模型1(MNIST FOR EXPERT)训练准确率曲线}
  \end{figure}

最终准确率:99.143\%

\item[(2)]模型2(LeNet-5):

训练曲线如下:

 \begin{figure}[H]
  \centering\includegraphics[width=4.5in]{./3.png}
  \caption{模型2(LeNet-5)训练准确率曲线}
  \end{figure}

最终准确率:98.198\%

\end{description}

由训练准确率曲线可以清楚地看出,模型1不仅收敛更快,而且最终准确率更高,实际测试中,模型1训练时间也较短,故以下各测试均基于模型1进行测试。


\subsection{卷积层层数的选择}

以下测试均基于模型1(MNIST FOR EXPERT),仅改变卷积层层数,训练次数为50000次。

\begin{description}

\item[(1)]卷积层层数为1时,训练10000步时准确率为98.4733\%,最终准确率99.0458\%。

\item[(2)]卷积层层数为2时,即为原始模型,训练10000步时准确率为99.2366\%,最终准确率99.8091\%。
\item[(3)]
卷积层层数为3时,训练10000步时准确率为99.0458\%,最终准确率99.4275\%。

\end{description}
三者准确率随训练次数收敛速度近乎相等,但是训练速度1层快于2层快于3层,消耗资源1层少于2层少于3层。

可见,卷积层层数为2为最合适的参数。层数减小训练速度加快、消耗资源减少,但准确率不高;层数增加训练速度减慢,消耗资源增加,而且准确率也并没有提高。

全连接层一般也选择2-3层,减少或增加层数可预测结果应该类似,由于时间有限,不再进行实验。

\subsection{训练集的选择}

以下测试均基于模型1(MNIST FOR EXPERT),卷积层层数为2,仅改变训练集,训练次数为50000次。

\begin{description}

\item[(1)]采用原始训练集时,验证集准确率为99.8071\%,Kaggle测试最终准确率为99.143\%,排名约为350名。

\item[(2)]采用MNIST训练集,验证集准确率为99.6183\%,Kaggle测试最终准确率为99.871\%,排名为52名。

\end{description}
\subsection{最终模型与竞赛成绩}

最终采用模型1(MNIST FOR EXPERT),卷积层层数为2,训练次数为50000次,两种训练集结果如图所示:

 \begin{figure}[H]
  \centering\includegraphics[width=4.5in]{./rank.png}
  \caption{排名和提交次数}
  \centering\includegraphics[width=4.5in]{./data.png}
  \caption{不同数据集训练下的准确率}
  \end{figure}

准确率最高时提交次数:3次

最终提交次数:4次


\section{实验心得}
  通过本次实验,我大大加深了对卷积神经网络的理解,初步掌握了神经网络的参数选择技巧,也初步学习了Tensorflow框架的使用方法,进入了高级机器学习的新世界。同时也学习了LaTex排版技巧。

  在这次实验中,我充分感受到了神经网络的神奇之处,通过这样的连接,识别准确率竟然可以达到99.871\%,接近100\%,而且是在噪声很高的数据集上进行的测试,实际表现应该会更好。

  但是这只是一个入门,神经网络的高级技巧和参数设置技巧我还没有完全掌握,还需要进一步的练习。

  值得思考的是,LeNet-5是上课讲的专门针对手写字母识别的卷积神经网络,我本以为效果会很好,结果却并不够理想,可能是卷积核和神经元太少,没有充分提取图像的特征。

  另外遗憾的是,由于期末复习时间太紧张,没有时间学习其余的神经网络,也没有时间对比训练结果,比如LSTM,希望暑假有机会可以进一步尝试。

\section{后期工作}
  我认为可以通过以下方式进行改进,留给以后的工作。

  \begin{description}
    \item[(1)]对训练集进行处理,例如适当旋转一定角度,适当放大或缩小,从而获得更多的训练数据,使得训练更为充分。

    \item[(2)]对测试集进行处理,例如去噪声,比如有的测试数据有一个黑点,可以采用深度优先搜索和设置阈值的方法得以实现。

    \item[(3)]采用RNN、LSTM等不同的神经网络进行训练与测试,对比结果。
  \end{description}

\section{源代码}
  运行方式:

  将文件train.csv和test.cs拷贝至Python程序所在目录,在命令行中输入python main.py即可。

  输出结果为文件result.csv,同时会保存训练出的模型,可以直接使用Tensorflow中的模块读取训练好的模型。

\lstinputlisting[language=python]{./main.py}

\begin{thebibliography}{9}
\bibitem{人工智能导论}
  人工智能导论,
  \emph{马少平},
  2017.
\bibitem{Python机器学习及实践:从零开始通往Kaggle竞赛之路}
  Python机器学习及实践:从零开始通往Kaggle竞赛之路,
  \emph{范淼, 李超},
  2017.
\end{thebibliography}

\end{document}
